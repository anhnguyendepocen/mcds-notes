\chapter{Survival Trees}


%%%%%%%%%%%%%%%%%%%%%%%%%%%%%%%%%%%%%%%%%%%%%%%%%%%%%%%%%%%%%%%%%%%%%%%%%%%%%%%%%%

\section{The Log-Rank Test}

The \textbf{log-rank test} (Mantel 1966; Peto and Peto 1972) is a test for statistical equivalence of two survival curves. It is obtained by constructing a $2x2$ contingency table at the time of each event and comparing the failure rates between the two groups, conditional on the number at risk in each group\footnote{See https://bookdown.org/sestelo/sa\_financial/comparing-survival-curves.html.}. In this way, the test compares the entire survival experience between groups. The null hypothesis is that the true underlying curves for the two groups are identical.

``In the absence of censoring, these methods reduce to the Wilcoxon-Mann-Whitney rank-sum test (Mann and Whitney 1947) for two samples and to the Kruskal-Wallis test (Kruskal and Wallis 1952) for more than two groups of survival times.''



%%%%%%%%%%%%%%%%%%%%%%%%%%%%%%%%%%%%%%%%%%%%%%%%%%%%%%%%%%%%%%%%%%%%%%%%%%%%%%%%%%

\section{Survival Example: Primary Biliary Cirrhosis}

Now let's consider how the same tree-building machinery 

This data is from the Mayo Clinic trial in primary biliary cirrhosis (PBC) of the liver conducted between 1974 and 1984. A total of 424 PBC patients, referred to Mayo Clinic during that ten-year interval, met eligibility criteria for the randomized placebo controlled trial of the drug D-penicillamine. The first 312 cases in the data set participated in the randomized trial and contain largely complete data. The additional 112 cases did not participate in the clinical trial, but consented to have basic measurements recorded and to be followed for survival. Six of those cases were lost to follow-up shortly after diagnosis, so the data here are on an additional 106 cases as well as the 312 randomized participants.



\begin{question}{}
Here's a dataset of survival data... how would you build a survival forest?
\end{question}

%%%%%%%%%%%%%%%%%%%%%%%%%%%%%%%%%%%%%%%%%%%%%%%%%%%%%%%%%%%%%%%%%%%%%%%%%%%%%%%%%%

\section{Random Survival Forests}

%%%%%%%%%%%%%%%%%%%%%%%%%%%%%%%%%%%%%%%%%%%%%%%%%%%%%%%%%%%%%%%%%%%%%%%%%%%%%%%%%%

\section{Boosted Survival Trees}

